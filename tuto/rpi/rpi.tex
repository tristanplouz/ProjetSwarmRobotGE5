\documentclass[a4paper, 10pt]{article}

\usepackage[T1]{fontenc}
\usepackage[utf8]{inputenc}
\usepackage[french]{babel}

\usepackage[top=20mm, bottom=20mm, left=10mm, right=10mm]{geometry}

\usepackage{hyperref}

\usepackage{mathpazo}
\usepackage{xcolor}
\definecolor{raspberry}{rgb}{0.737,0.067,0.259}

\usepackage{fancyhdr}
\pagestyle{fancy}

\usepackage[ddmmyyyy]{datetime}
\renewcommand{\dateseparator}{/}

\usepackage{listings}
\usepackage{tcolorbox}
\tcbuselibrary{listings}

\newtcblisting{commandshell}{
	colback=black,
	colupper=white,
	colframe=orange!75!white,
	listing only,
	listing options={style=tcblatex,language=sh},
	every listing line={
		\textcolor{green}{\small\ttfamily\bfseries pi@raspberrypi}:\textcolor{blue}{\small\ttfamily\bfseries \~\ \$\ }
	}
}

\fancyhead[L]{Version du 26/09/2020,\\ modifié le \today}
\fancyhead[C]{\textit{Premiers pas avec Raspberry Pi}}
\fancyhead[R]{Swarm Drone Flight}
\fancyfoot[L]{Florian Pouthier}
\fancyfoot[R]{\jobname.pdf}
\headheight = 15pt

\begin{document}

\begin{center}
\LARGE{\textbf{Premiers pas avec Raspberry Pi}}
\end{center}

Le projet \textit{Swarm Drone Flight} nécessite l'utilisation d'un
\textbf{Raspberry Pi} embarqué sur le drone \textit{master} de l'essaim.
Ce tutoriel est là pour lister les premières étapes de prise en main
du micro-ordinateur qui agit comme le \textit{"cerveau"} de l'essaim
de drones. 

\section{Faire une image de l'OS de la RPi}

La première étape est de faire une image d'OS du RPi sur une
carte SD. Le logiciel utilisé ici est \textbf{Raspberry Pi Imager}.

\begin{enumerate}
	\item\textbf{Insérer la carte SD dans le PC.} La carte SD insérée peut
	être vierge ou non, dans tous les cas le support sera formaté au cours
	du processus. 
	
	\item\textbf{Lancer Pi Imager.} La fenêtre principale du logiciel 
	devrait ressembler à la capture de la \textsc{Figure \ref{sc_pi_imager}}.
	
	\begin{figure}[h]
		\centering
		\includegraphics[scale=0.4]{fig/sc_pi_imager.png}
		\caption{Fenêtre principale de Raspberry Pi Imager.}
		\label{sc_pi_imager}
	\end{figure}

	\item\textbf{Choisir l'OS du RPi.} Nous allons installer un OS Lite,
	qui sera largement suffisant pour ce projet. Cliquer sur \texttt{CHOOSE OS}, 
	puis aller dans la liste \texttt{Raspberry Pi OS (other)}. 
	Dans cette liste, sélectionner \texttt{Raspberry Pi OS Lite (32-bit)}.
	
	\item\textbf{Choisir la carte SD.} Cliquer sur \texttt{CHOOSE SD CARD}, puis
	sur le périphérique SD qui devrait apparaître dans la liste.
	
	\item\textbf{Écrire la carte SD.} Cette fois-ci la fenêtre principale de 
	Pi Imager devrait ressembler à la capture de la \textsc{Figure \ref{write_pi_imager}}.
	
	\begin{figure}[h]
		\centering
		\includegraphics[scale=0.4]{fig/write_pi_imager.png}
		\caption{Fenêtre de Raspberry Pi Imager après configuration.}
		\label{write_pi_imager}
	\end{figure}
	
	Le bouton \texttt{WRITE} devient à ce moment là cliquable. Cliquer sur 
	\texttt{WRITE}, puis cliquer sur \texttt{YES} pour valider le formatage
	de la carte SD. L'écriture de la carte SD se lance alors. 
	Une fois l'écriture terminée, cliquer sur \texttt{CONTINUE}. 
	Vous pouvez maintenant retirer la carte SD de votre PC et l'insérer
	dans votre Raspberry Pi.
\end{enumerate}

\vspace{-3em}

\section{Premier démarrage du RPi}

\vspace{-0.5em}

Il est maintenant temps de travailler avec votre Raspberry Pi !

\begin{enumerate}
	\item\textbf{Brancher tous les périphériques nécessaires.} Brancher dans
	un premier temps tous les périphériques qui seront utiles pour le
	premier démarrage de votre RPi :
	
	\begin{itemize}
		\vspace{0.5em}
		\item[$\bullet$] \textbf{Ports USB :} un clavier, si besoin une souris
		mais ce n'est pas vraiment nécessaire pour un OS Lite.
		\vspace{0.5em}
		\item[$\bullet$] \textbf{HDMI :} un écran, un adaptateur VGA/HDMI peut
		s'avérer nécessaire si l'écran ne dispose que d'une connectique VGA.
		\vspace{0.5em}
		\item[$\bullet$] \textbf{Ethernet :} Si vous ne souhaitez pas utiliser
		le Wi-Fi vous pouvez vous connecter à Internet via un câble ethernet.
		Dans tous les cas, une connexion internet sera requise pour suivre ce
		tuto en intégralité.
		\vspace{0.5em}
	\end{itemize}
	
	\item\textbf{Insérer la carte SD sur le RPI.} A faire avant de mettre le
	micro-ordinateur sous tension !
	
	\item\textbf{Mettre le RPI sous tension.} Le RPi est à alimenter via le
	connecteur micro-USB. Il faut avoir un adaptateur secteur pouvant délivrer
	au minimum 2.5A, au risque de sous-alimenter le micro-ordinateur.
\end{enumerate}

D'autres informations utiles sur la prise en main du Raspberry Pi sont 
disponibles à cette adresse :
\url{https://projects.raspberrypi.org/en/projects/raspberry-pi-getting-started/}

\vspace{-1em}

\section{Première connexion au RPi}

\vspace{-0.5em}

\textit{Si vous êtes déjà familier avec les environnements Linux, vous 
verrez que la plupart des commandes sont celles utilisées 
dans un terminal Linux.}

\begin{enumerate}
	\item\textbf{Premier login.} Une fois le micro-ordinateur sous tension, 
	vous verrez à l'écran le démarrage de tous les services du RPi. 
	Une fois tous les services chargés, vous serez invités à renseigner un 
	login et un mot de passe. Les logins par défaut sont:
	
	\textbf{raspberrypi login:} \texttt{pi}

	\textbf{Password:} \texttt{raspberry}
	Il peut s'agir d'un grave problème sécurité pour une application sensible, 
	en effet, en conservant les informations par défaut, tout le monde peut les 
	utiliser pour se connecter. Il faut donc penser à les changer. Cela peut se 
	faire en utilisant la commande \texttt{raspi-config}, vue plus loin, et le 
	menu \texttt{change\_pass}.
	Le clavier étant par défaut paramétré en QWERTY, la tentative de connexion 
	devrait échouer. En effet, il faudra saisir \texttt{rqspberry} pour pouvoir 
	réussir à se logger.
	
	\item\textbf{Changer le clavier en AZERTY.} On va tout de suite régler 
	ce problème de clavier pour terminer sereinement le tuto. 
	Taper la commande suivante pour accéder au menu de configuration du RPi :
	
\begin{commandshell}
sudo raspi-config
\end{commandshell}
	
	\textit{(Saisir en réalité}\
	\texttt{sudo rqspi-config}\
	\textit{pour que ça fonctionne).}
	
	Dans le menu de configuration, sélectionner successivement :

	\begin{itemize}
		\item[$\bullet$] \texttt{4 Localisation Options} 
						 $\rightarrow$ \textbf{ENTER}
		\item[$\bullet$] \texttt{I3 Change Keyboard Layout}
						 $\rightarrow$ \textbf{ENTER}
		\item[$\bullet$] \texttt{Generic 105-Key PC (int1.)}
						 $\rightarrow$ \textbf{ENTER}
		\item[$\bullet$] \texttt{Other}
						 $\rightarrow$ \textbf{ENTER}
		\item[$\bullet$] \texttt{French}
						 $\rightarrow$ \textbf{ENTER}	
		\item[$\bullet$] \texttt{French}
						 $\rightarrow$ \textbf{ENTER}	
		\item[$\bullet$] \texttt{The default for the keyboard layout}
						 $\rightarrow$ \textbf{ENTER}	
		\item[$\bullet$] \texttt{No compose key}
						 $\rightarrow$ \textbf{ENTER}			 		
	\end{itemize}
	
	Vous revenez finalement au menu principal de configuration.
	
	\item\textbf{Configurer le Wi-Fi.} Si vous n'avez pas de connexion
	filaire, vous devez configurer une connexion au réseau Wi-Fi pour
	pouvoir accéder à internet. Pour cela, toujours dans le menu de
	configuration, sélectionner successivement :

	\begin{itemize}
		\item[$\bullet$] \texttt{2 Network Options} 
						 $\rightarrow$ \textbf{ENTER}
		\item[$\bullet$] \texttt{N2 Wireless LAN}
						 $\rightarrow$ \textbf{ENTER}
		\item[$\bullet$] \texttt{FR France}
						 $\rightarrow$ \textbf{ENTER}
		\item[$\bullet$] \texttt{<OK>}
						 $\rightarrow$ \textbf{ENTER}
		\item[$\bullet$] \texttt{Please enter SSID}
						 $\rightarrow$ Saisir le nom du réseau Wi-Fi
						 $\rightarrow$ \textbf{ENTER}	
		\item[$\bullet$] \texttt{Please enter passphrase}
						 $\rightarrow$ Saisir le mot de passe du Wi-Fi
						 $\rightarrow$ \textbf{ENTER}		 		
	\end{itemize}
	
	Vous revenez finalement au menu principal de configuration.
	Pour sortir du menu, sélectionner \texttt{<Finish>}
	$\rightarrow$ \textbf{ENTER}.	

	\item\textbf{Redémarrer le RPi.} Pour prendre en compte les 
	différentes modifications précédentes, redémarrer le RPi avec
	la commande suivante :	
	
\begin{commandshell}
sudo reboot
\end{commandshell}	

	Ressaisir finalement le login \texttt{pi} et le mot de passe \texttt{raspberry} ou votre nouveau mot de passe. 
\end{enumerate}

\section{Installation des packages}

Pour travailler sur le projet \textit{Swarm Drone Flight}, il faut
installer certains packages qui ne sont pas initialement installés
dans l'OS Lite.

\begin{enumerate}
	\item\textbf{Vérifier la connexion internet.} La connexion internet 
	peut être rapidement vérifiée en faisant un \texttt{ping} sur un
	site web quelconque.
	
\begin{commandshell}
ping google.fr
\end{commandshell}

	Si des lignes de réception de paquets apparaissent, alors la
	connexion internet est bien établie ! Sinon vérifier que les
	login et mot de passe du réseau Wi-Fi ciblé sont bien paramétrés
	en refaisant la procédure décrite à la page précédente.	
	Arrêter le processus avec \textbf{Ctrl} + \textbf{C}.
	
	\item\textbf{Mettre à jour l'OS.} Afin d'obtenir la dernière version
	de l'OS et des paquets logiciels, saisir :
	
\begin{commandshell}
sudo apt update && sudo apt upgrade
\end{commandshell}
	
	Un message demandant si vous autorisez le téléchargement de paquets
	devrait également apparaître, valider en appuyant \textbf{ENTER}.
	
	\item\textbf{Installer le gestionnaire de version}. Installer \texttt{Git} pour le 
	\textit{versionning} des codes utilisés en saisissant :
	
\begin{commandshell}
sudo apt-get install git
\end{commandshell}	
	
	\item\textbf{Installer Python 3}. Afin d'installer \texttt{Python 3},
	saisir :
	
\begin{commandshell}
sudo apt-get install python3
\end{commandshell}

	\item\textbf{Charger l'installateur de paquets Python 3}. 
	L'installateur de paquets \texttt{Python 3} peut être installé avec :
	
\begin{commandshell}
sudo apt-get install python3-pip
\end{commandshell}
  	\item\textbf{Installer le module de communication Série}
  	Afin de pouvoir communiquer sur une liaison série, avec le module \texttt{DroneKit},
  	il faut installer le module python adéquat:
  	\begin{commandshell}
sudo apt-get install python3-serial
\end{commandshell}

	On peut installer tous les dépendances nécessaire en une commande:
\begin{commandshell}
sudo apt update && sudo apt upgrade && sudo apt-get install git python3 python3-pip python3-serial
\end{commandshell}

	\item\textbf{Installer le module \texttt{dronekit}}. 
	Le module python \texttt{dronekit} permet la communication avec un
	contrôleur de vol (un \textit{Pixhawk} par exemple) en 
	utilisant un protocole MAVLink.
	
\begin{commandshell}
pip3 install dronekit
\end{commandshell}
\end{enumerate}

\section{Premiers tests}
\begin{enumerate}

	\item\textbf{Cloner le dépot de code}.
	Maintenant que toutes les dépendances sont installées, on peut charger les différents codes nécessaire.
	Pour cela, on va cloner le dépot \texttt{github} associé au projet avec tous les codes.
\begin{commandshell}
git clone https://github.com/tristanplouz/ProjetSwarmRobotGE5.git
\end{commandshell}
\end{enumerate}

\end{document}